\documentclass[a4paper,11pt]{article}

\usepackage[toc,page]{appendix}
\usepackage{amsthm}
\usepackage{amsmath}
\usepackage{latexsym}
\usepackage{amssymb}
\usepackage{verbatim}
\usepackage{setspace}
\usepackage{qtree}
\usepackage{enumitem}
\usepackage{xcolor}
\usepackage{url}
\usepackage{stmaryrd}
\usepackage{authblk}
\usepackage[margin=1.4in]{geometry}

\author{Reijo Jaakkola}

\begin{document}

\setlength\abovedisplayskip{3pt}
\setlength\belowdisplayskip{3pt}

\title{Encoding NP-hard  problems}

\theoremstyle{plain}
\newtheorem{theorem}{Theorem}[section]
\newtheorem{lemma}[theorem]{Lemma}
\newtheorem{corollary}[theorem]{Corollary}
\newtheorem{proposition}[theorem]{Proposition}
\theoremstyle{definition}
\newtheorem{definition}[theorem]{Definition}
\newtheorem{remark}[theorem]{Remark}
\newtheorem{example}[theorem]{Example}

\newcommand{\PFO}{\mathrm{FO}^+}

\maketitle


The purpose of this note is to give several reductions from NP-hard graph problems to the validity problem of positive first-order logic $\PFO$. We will start by giving reductions for the following
problems.
\begin{enumerate}
	\item Hamiltonian circuit. The task is to determine whether a given graph contains a cycle that "visits" each vertex exactly ones.
           \item Clique set. The task is to determine whether a given graph contains a set of $k$-vertices so that every two members of the set are adjacent.
           \item Vertex cover. The task is to determine whether a given graph contains a set of $k$-vertices so that for every edge of the graph, at least one of its endpoints belongs to this set.
\end{enumerate}
First we will introduce some notation. Let $G = (V,E)$ be the input graph, where $V = \{v_1,...,v_n\}$. We will use $\Sigma(G)$ to denote
the following set of atomic formulas.
\[\{c_i \neq c_j \mid 1\leq i < j \leq n\} \cup \{c_i E c_j \mid (v_i,v_j) \in E\} \cup \{\neg c_i E c_j \mid (c_i,c_j) \not\in E\}.\]
Notice that we are using $E$ to denote both a binary relational symbol $E$ as well as the edge set. Now we will associate to each of the above problems a sentence $\phi$
of $\PFO$ with the property that $\Sigma(G)\models \phi$ if and only if $\Sigma(G)$ is a positive instance of the problem.
\begin{enumerate}
	\item In the case of Hamiltonian circuit the sentence is
	\[\exists x_1 ... \exists x_n (\bigwedge_{1\leq i\leq n} \bigvee_{1\leq j\leq n} c_i = x_j \land \bigwedge_{1\leq i < n} x_i E x_{i+1} \land x_n E x_1)\]
	\item In the case of clique set the sentence is
	\[\exists x_1 ... \exists x_k \bigwedge_{1\leq i \leq k} \bigwedge_{1\leq j\leq k} x_i E x_j\]
	\item In the case of vertex cover the sentence is
	\[\exists x_1 ... \exists x_k (\bigwedge_{1\leq \ell \leq k} \bigvee_{1\leq i \leq n} x_\ell = c_i \land \bigwedge_{c_i E c_j} \bigvee_{1\leq \ell \leq k} (x_\ell = c_i \lor x_\ell = c_j))\]
\end{enumerate}

\end{document}










